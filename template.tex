\newcommand{\drawtop}[9]{
	\coordinate (v0) at (0, 1);
	\coordinate (v1) at (-0.288675, 0.833333);
	\coordinate (v2) at (0, 0.666667);
	\coordinate (v3) at (0.288675, 0.833333);
	\coordinate (v4) at (-0.577350, 0.666667);
	\coordinate (v5) at (-0.288675, 0.5);
	\coordinate (v6) at (0, 0.333333);
	\coordinate (v7) at (0.288675, 0.5);
	\coordinate (v8) at (0.577350, 0.666667);
	\coordinate (v9) at (-0.866025, 0.5);
	\coordinate (v10) at (-0.577350, 0.333333);
	\coordinate (v11) at (-0.288675, 0.166667);
	\coordinate (v12) at (0.288675, 0.166667);
	\coordinate (v13) at (0.577350, 0.333333);
	\coordinate (v14) at (0.866025, 0.5);
	\coordinate (v15) at (0, 0);

	\fill[color = #1] (v0) -- (v1) -- (v2) -- (v3);
	\fill[color = #2] (v1) -- (v4) -- (v5) -- (v2);
	\fill[color = #3] (v4) -- (v9) -- (v10) -- (v5);
	\fill[color = #4] (v10) -- (v11) -- (v6) -- (v5);
	\fill[color = #5] (v5) -- (v6) -- (v7) -- (v2);
	\fill[color = #6] (v2) -- (v7) -- (v8) -- (v3);
	\fill[color = #7] (v7) -- (v13) -- (v14) -- (v8);
	\fill[color = #8] (v6) -- (v12) -- (v13) -- (v7);
	\fill[color = #9] (v11) -- (v15) -- (v12) -- (v6);

	%Left-Up diagonal lines
	\draw[color = black] (v9) -- (v0);
	\draw[color = black] (v10) -- (v3);
	\draw[color = black] (v11) -- (v8);
	\draw[color = black] (v15) -- (v14);

	%Down-Right diagonal lines
	\draw[color = black] (v0) -- (v14);
	\draw[color = black] (v1) -- (v13);
	\draw[color = black] (v4) -- (v12);
	\draw[color = black] (v9) -- (v15);
}

\newcommand{\drawleft}[9]{
	\coordinate (v0) at (-0.866025, 0.5);
	\coordinate (v1) at (-0.577350, 0.333333);
	\coordinate (v2) at (-0.288675, 0.166667);
	\coordinate (v3) at (0, 0);
	\coordinate (v4) at (-0.866025, 0.166667);
	\coordinate (v5) at (-0.577350, 0.0);
	\coordinate (v6) at (-0.288675, -0.166667);
	\coordinate (v7) at (0, -0.333333);
	\coordinate (v8) at (-0.866025, -0.166667);
	\coordinate (v9) at (-0.577350, -0.333333);
	\coordinate (v10) at (-0.288675, -0.5);
	\coordinate (v11) at (0, -0.666667);
	\coordinate (v12) at (-0.866025, -0.5);
	\coordinate (v13) at (-0.577350, -0.666667);
	\coordinate (v14) at (-0.288675, -0.833333);
	\coordinate (v15) at (0, -1);

	\fill[color = #1] (v0) -- (v1) -- (v5) -- (v4);
	\fill[color = #2] (v1) -- (v2) -- (v6) -- (v5);
	\fill[color = #3] (v2) -- (v3) -- (v7) -- (v6);
	\fill[color = #4] (v4) -- (v5) -- (v9) -- (v8);
	\fill[color = #5] (v5) -- (v6) -- (v10) -- (v9);
	\fill[color = #6] (v6) -- (v7) -- (v11) -- (v10);
	\fill[color = #7] (v8) -- (v9) -- (v13) -- (v12);
	\fill[color = #8] (v9) -- (v10) -- (v14) -- (v13);
	\fill[color = #9] (v10) -- (v11) -- (v15) -- (v14);

	%Down-Right diagonal lines
	\draw[color = black] (v0) -- (v3);
	\draw[color = black] (v4) -- (v7);
	\draw[color = black] (v8) -- (v11);
	\draw[color = black] (v12) -- (v15);

	%Vertical lines
	\draw[color = black] (v0) -- (v12);
	\draw[color = black] (v1) -- (v13);
	\draw[color = black] (v2) -- (v14);
	\draw[color = black] (v3) -- (v15);
}

\newcommand{\drawright}[9]{
	\coordinate (v0) at (0, 0);
	\coordinate (v1) at (0.288675, 0.166667);
	\coordinate (v2) at (0.577350, 0.333333);
	\coordinate (v3) at (0.866025, 0.5);
	\coordinate (v4) at (0, -0.333333);
	\coordinate (v5) at (0.288675, -0.166667);
	\coordinate (v6) at (0.577359, 0.0);
	\coordinate (v7) at (0.866025, 0.166667);
	\coordinate (v8) at (0, -0.666667);
	\coordinate (v9) at (0.288675, -0.5);
	\coordinate (v10) at (0.577359, -0.333333);
	\coordinate (v11) at (0.866025, -0.166667);
	\coordinate (v12) at (0, -1);
	\coordinate (v13) at (0.288675, -0.833333);
	\coordinate (v14) at (0.577359, -0.666667);
	\coordinate (v15) at (0.866025, -0.5);

	\fill[color = #1] (v0) -- (v1) -- (v5) -- (v4);
	\fill[color = #2] (v1) -- (v2) -- (v6) -- (v5);
	\fill[color = #3] (v2) -- (v3) -- (v7) -- (v6);
	\fill[color = #4] (v4) -- (v5) -- (v9) -- (v8);
	\fill[color = #5] (v5) -- (v6) -- (v10) -- (v9);
	\fill[color = #6] (v6) -- (v7) -- (v11) -- (v10);
	\fill[color = #7] (v8) -- (v9) -- (v13) -- (v12);
	\fill[color = #8] (v9) -- (v10) -- (v14) -- (v13);
	\fill[color = #9] (v10) -- (v11) -- (v15) -- (v14);

	%Up-Right diagonal lines
	\draw[color = black] (v0) -- (v3);
	\draw[color = black] (v4) -- (v7);
	\draw[color = black] (v8) -- (v11);
	\draw[color = black] (v12) -- (v15);

	%Vertical lines
	\draw[color = black] (v0) -- (v12);
	\draw[color = black] (v1) -- (v13);
	\draw[color = black] (v2) -- (v14);
	\draw[color = black] (v3) -- (v15);
}
